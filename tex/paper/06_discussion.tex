%We have shown that \sysname can be used to detect the insecure crypto keys in real world binary executables. 
%However, 
\sysname %is still not perfect and it 
has a number of limitations. %In this section, we discuss the limitations of \sysname and outline our future work.
%
%
First, while \sysname is able to pinpoint the insecure keys, it does not report any specific crypto algorithms (e.g,. \textsf{\small AES}, \textsf{\small DSA}) to which the insecure key belongs. 
In fact, \sysname has all the building blocks to support the identification of each specific crypto algorithm used by an binary executable. 
More specifically, since \sysname has identified the data bundles of key buffer $K$, ciphertext $C$, and plaintext $P$, we can actually perform a brute force search of the encryption algorithm $E$ by computing whether $C = E (K, P)$, where $E$ are those well-known crypto algorithms. 
Certainly, this is only possible when software uses standard crypto algorithms (e.g., if they follow the never-implement-your-own-crypto practice~\cite{schneier1999cryptography, apvrille2005secure}). 
We leave the identification of specific crypto algorithms in a binary executable as one of our future efforts.

Second, \sysname performs the taint propagation at the function level. 
That is, if a function uses a tainted tag, all the data defined in that function will have that tainted tag. 
Such taint propagation may overly propagate the tainted tag, making insecure keys appear secure. 
For instance, it might be possible that a function uses a random function, but the return value of the random function is never assigned to the crypto key. 
While we have not encountered such a case, we plan to address this issue by implementing a fine-grained taint propagation policy, and meanwhile address the performance issues caused from this policy.

Finally, \sysname will not be able to detect the crypto keys if they are stored in CPU registers. 
A particular case is the secure in-cache execution~\cite{guan2014copker, chen2017secure} technique against the cold-boot attack~\cite{halderman2009lest}.
In this case the crypto key is never evicted to memory and thus our approach is not able to detect it.

There are also other possible extensions of \sysname. 
For instance, crypto operations are increasingly used by mobile apps to protect their sensitive data. Thus, extending \sysname to detect insecure keys in mobile apps is a logical next step. 

